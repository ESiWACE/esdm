\section{Building ESDM documentation}

\begin{figure}[!ht]
  \centering
  \includegraphics[width=\textwidth]{../assets/build-docs-overview/graph.png}
  \caption{Conversion paths.}
  \label{fig:build-docs-overview}
\end{figure}

The documentation is organized as a set of latex and resource files.
This structure simplifies integration in other documents, if only particular parts of the documentation are required.
The source files are located in \lstinline|./doc/latex|.
This is a central location, where files can be modified, if changes are required.
All other locations listed in this sections are auto-generated.
Therefore, after recompilation all changes will be overwritten.

Documentation centralization reduces documentation efforts.
As shown in \Cref{fig:build-docs-overview} the latex source files are exported to different formats.
Change in the latex files will affect all documents.

PDF and Github-Markdown format don't include API reference.
You find API reference in the Latex and HTML formats generated by \lstinline|doxygen|.

The listings in this section assume that current working directory is the ESDM repository.
The subsections discuss the supported export possibilities in detail.

\subsection{PDF}
Latex documentation can be compiled by commands shown in \Cref{lst:doc:latex}.
The result will be stored in \lstinline|./doc/latex/main.pdf|

\begin{lstlisting}[caption={Make PDF document},label={lst:doc:latex}]
cd ./doc
make
\end{lstlisting}

\subsection{Doxygen}
Doxygen depends a set of auto-generated markdown files, which should never be modified manually, because all changes will be overwritten by next compilation.
Please always work with files in \lstinline|./doc/latex| folder and compile them running the make script as shown in \Cref{lst:dox:doxy-markdown}.

\begin{lstlisting}[caption={Markdown files generation},label={lst:dox:doxy-markdown}]
cd ./doc
make
\end{lstlisting}

The resulting \lstinline|*.md| files are generated in \lstinline|./doc/markdown| directory.
Now all required source files for doxygen should be available and the final documentation can compiled by \lstinline|doxygen| as shown in \Cref{lst:doc:doxygen}.

\begin{lstlisting}[caption={Make HTML and Latex documents},label={lst:doc:doxygen}]
./configure
cd build
doxygen
\end{lstlisting}

The resulting HTML start page is located in \lstinline|./build/doc/html/index.html| and main Latex document is located in \lstinline|./build/doc/latex/refman.tex|.

\subsection{Github-Markdown}
The Github documentation \lstinline|./README.md| is generated by \lstinline|pandoc| from the \lstinline|./doc/latex/main.tex| file.
%\lstinline|pandoc| uses a lua-filter (\lstinline|./doc/latex/image-path-filter.lua|), to change the paths to resource files.
\lstinline|./README.md| shold be never modified manually, since changes will be overwritten by the next documentation compilation.
The compilation commands are shown in \Cref{lst:doc:gfm}.

\begin{lstlisting}[caption={Make ./README.md},label={lst:doc:gfm}]
cd ./doc
make
\end{lstlisting}
