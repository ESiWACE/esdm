\section{Building instructions}%
\label{installation-instructions-for-mistral}
%Mistral (HLRE-3) is the supercomputer installed at DKRZ in 2015/2016.
This guide documents installation procedures to build prerequisites as well as the prototype codebase for development and testing purposes.

\begin{figure}[!ht]
  \begin{center}
    \includegraphics[width=0.6\textwidth]{../assets/deps/graph.png}
  \end{center}
  \caption{Dependencies overview}
  \label{fig:deps}
\end{figure}

\subsection{Dependencies (Spack)}%
\label{satisfying-requirements}
Although the software trees on supercomputer may provide all the standard tools and libraries required to build ESDM, it is still recommended to build, install, and manage dependencies via Spack\footnote{\url{https://spack.readthedocs.io/en/latest/index.html}}.
The instructions below show how to setup Spack and install the ESDM dependencies.

\begin{enumerate}
  \item Download and enable Spack
    \begin{lstlisting}
    git clone --depth=2 https://github.com/spack/spack.git spack
    . spack/share/spack/setup-env.sh
    \end{lstlisting}
  \item Set a \lstinline|gcc| version to be used
    \begin{lstlisting}
    gccver=7.3.0
    \end{lstlisting}
  \item Install dependencies
    \begin{lstlisting}
    spack install gcc@$gccver +binutils
    spack compiler find
    spack install autoconf%gcc@$gccver
    spack install openmpi%gcc@$gccver gettext%gcc@$gccver
    spack install jansson%gcc@$gccver glib%gcc@$gccver
    \end{lstlisting}
  \item Load dependencies
    \begin{lstlisting}
    spack load gcc@$gccver
    spack load -r autoconf%gcc@$gccver
    spack load -r libtool%gcc@$gccver
    spack load -r openmpi%gcc@$gccver
    spack load -r jansson%gcc@$gccver
    spack load -r glib%gcc@$gccver
    \end{lstlisting}
\end{enumerate}


\subsection{ESDM prototype}
Assuming all prerequisites have been installed and tested, ESDM can be configured and build as follows.
\begin{enumerate}
  \item Ensure environment is aware of dependencies installed using \lstinline|spack| and \lstinline|dev-env|
    \begin{lstlisting}
    git clone https://github.com/ESiWACE/esdm
    \end{lstlisting}
  \item Configure and build ESDM
    \begin{lstlisting}
    cd esdm
    pushd deps
    ./prepare.sh
    popd
    ./configure \
      --prefix=${PREFIX} \
      --enable-netcdf4
    cd build
    make
    make install
    \end{lstlisting}
\end{enumerate}


\subsection{ESDM-NetCDF and ESDM-NetCDF-Python}%
\begin{enumerate}
  \item Clone the NetCDF-ESDM repository
    \begin{lstlisting}
    git clone https://github.com/ESiWACE/esdm-netcdf-c
    \end{lstlisting}
  \item Configure and build NetCDF-ESDM. (\lstinline|$INSTPATH| is the installation path of ESDM.)
    \begin{lstlisting}
    cd esdm-netcdf-c
    /bootstrap
    ./configure \
      --prefix=$prefix \
      --with-esdm=$INSTPATH \
      LDFLAGS="-L$INSTPATH/lib" \
      CFLAGS="-I$INSTPATH/include" \
      CC=mpicc \
      --disable-dap
    make -j
    make install
    \end{lstlisting}
  \item If required, install the netcdf4-python module. 
    Change to the ESDM-NetCDF repository's root directory and install the patched NetCDF-Python module.
    \begin{lstlisting}
    cd dev
    git clone https://github.com/Unidata/netcdf4-python.git
    cd netcdf-python
    patch -s -p1 < ../v2.patch
    python3 setup.py install --user
    \end{lstlisting}
\end{enumerate}

\subsection{Docker}
For the development of ESDM, the directory \lstinline|./dev/docker| contains all requirements to set up a development environment using docker quickly.
The Docker files contain ESDM installation instructions for different distributions.
For easy building Docker files for different platforms are provided in different flavours:

\begin{itemize}
  \item CentOS/Fedora/RHEL like systems
  \item Ubuntu/Debian like systems
\end{itemize}

\subsubsection{Setup}

Assuming the docker service is running you can build the docker images as follows:

\begin{lstlisting}
cd ./dev/docker
cd <choose ubuntu/fedora/..>
sudo docker build -t esdm .
\end{lstlisting}

After docker is done building an image, you should see the ESMD test suite's output, which verifies that the development environment is set up correctly. 
The output should look similar to the following output:

\begin{lstlisting}
Running tests...
Test project /data/esdm/build
Start 1: metadata
1/2 Test #1: metadata .........................   Passed    0.00 sec
Start 2: readwrite
2/2 Test #2: readwrite ........................   Passed    0.00 sec

100% tests passed, 0 tests failed out of 2

Total Test time (real) =   0.01 sec
\end{lstlisting}


\subsubsection{Usage}
Running the ESDM Docker container will run the test suite:

\begin{lstlisting}
sudo docker run esdm  # should display test output
\end{lstlisting}

You can also explore the development environment interactively:

\begin{lstlisting}
sudo docker run -it esdm bash
\end{lstlisting}

\begin{lstlisting}
docker run dev/docker/ubuntu-whole-stack/Dockerfile
\end{lstlisting}
