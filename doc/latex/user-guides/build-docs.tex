\section{Documentation}

\begin{figure}[!ht]
  \begin{center}
    \includegraphics[width=\textwidth]{../assets/build-docs-overview/graph.png}
  \end{center}
  \caption{Conversion paths.}
  \label{fig:build-docs-overview}
\end{figure}

The documentation is organized as a set of latex and resource files.
This structure simplifies integration in other documents if only particular documentation parts are required.
The source files are located in \lstinline|./doc/latex|.
This is a central location where files can be modified if changes are required.
All other locations listed in these sections are auto-generated.
Therefore, after recompilation all changes will be overwritten.

Documentation centralization reduces documentation efforts.
Changes in the latex files will affect all other auto-generated documents.
As shown in \Cref{fig:build-docs-overview}, the latex source files are exported to PDF, Github-Markdown, Latex and HTML formats.
PDF and Github-Markdown don't include ESDM-API reference.
You find it in the Latex and HTML generated by Doxygen.

The \Cref{code:doc:latex,code:dox:doxy-markdown,code:doc:doxygen,code:doc:doxygen:pdf,code:doc:gfm} in this section assume that current working directory is the ESDM repository.
The subsections discuss the supported export possibilities in detail.

\subsection{PDF}
Latex documentation can be compiled to a PDF file by commands shown in \Cref{code:doc:latex}.
The result will be stored in \lstinline|./doc/latex/main.pdf|

\begin{lstlisting}[caption={Make PDF document},label={code:doc:latex}]
cd ./doc
make
\end{lstlisting}

\subsection{Github-Markdown}
The Github documentation \lstinline|./README.md| is generated by \lstinline|pandoc| from the \lstinline|./doc/latex/main.tex| file.
%\lstinline|pandoc| uses a lua-filter (\lstinline|./doc/latex/image-path-filter.lua|) to change the paths to resource files.
\lstinline|./README.md| should never be modified manually since the next documentation compilation will overwrite changes.
The compilation commands are shown in \Cref{code:doc:gfm}.

\begin{lstlisting}[caption={Make ./README.md},label={code:doc:gfm}]
cd ./doc
make
\end{lstlisting}

\subsection{HTML and Latex with API reference}
Doxygen depends on a set of auto-generated markdown files, which should never be modified manually because the next compilation will overwrite all changes.
Please always work with files in \lstinline|./doc/latex| folder and compile them running the make script as shown in \Cref{code:dox:doxy-markdown}.

\begin{lstlisting}[caption={Markdown files generation},label={code:dox:doxy-markdown}]
cd ./doc
make
\end{lstlisting}

The resulting \lstinline|*.md| files are generated in \lstinline|./doc/markdown| directory.
Now all required source files for Doxygen should be available, and the final documentation can be compiled by \lstinline|doxygen| as shown in \Cref{code:doc:doxygen}.

\begin{lstlisting}[caption={Make HTML and Latex documents},label={code:doc:doxygen}]
./configure
cd build
doxygen
\end{lstlisting}

The resulting HTML start page is located in \lstinline|./build/doc/html/index.html|, and the main Latex document is located in \lstinline|./build/doc/latex/refman.tex|.

\begin{lstlisting}[caption={Comple PDF with API reference},label={code:doc:doxygen:pdf}]
cd doc/latex
pdflatex refman.tex
\end{lstlisting}

The resulting PDF file is \lstinline|./build/doc/latex/refman.pdf|

