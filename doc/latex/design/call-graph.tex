\section{Callgraph for accessing metadata}%
\label{callgraph-for-accessing-metadata}

Guiding question:

\begin{itemize}
  \item How and when do we fetch metadata in the read path?
  \item When do we serialize metadata to JSON in the write path?
\end{itemize}

\subsection{General responsibility}%
\label{general-responsibility}

\begin{itemize}
  \item Container holds references to datasets (under different name possibly)
  \item Dataset holds information about the dataset itself:
    \begin{itemize}
      \item User-defined attributes (scientific metadata) =\textgreater{} it is not know a-priori what that is
      \item Technical attributes (some may be optional) =\textgreater{} well known how they look like
      \item Fragment information is directly inlined as part of datasets
    \end{itemize}
  \item Fragments
    \begin{itemize}
      \item Information about their shape, backend plugin ID, backend-specific options (unknown to us)
    \end{itemize}
\end{itemize}

\subsection{Write}%
\label{write}

Applies for datasets, containers, fragments:

\begin{itemize}
  \item All data is kept in appropriate structures in main memory
  \item Serialize to JSON just before calling the metadata backend to store the metadata (and free JSON afterwards)
  \item Backend communicates via JSON to ESDM layer
\end{itemize}

Callgraph from user perspective:

\begin{itemize}
  \item \lstinline|c = container_create("name")|
  \item \lstinline|d = dataset_create(c, "dset")|
  \item \lstinline|write(d)| creates fragments and attaches the metadata to the dataset
  \item \lstinline|dataset_commit(d)| =\textgreater{} make the dataset persistent, also write the dataset + fragment metadata
  \item \lstinline|container_commit(c)| =\textgreater{} makes the container persistent
    TODO: check that the right version of data is linked to it.
  \item \lstinline|dataset_destroy(d)|
  \item \lstinline|container_destroy(c)|
\end{itemize}

\subsection{Read}%
\label{read}

Applies for datasets, containers, fragments:

\begin{itemize}
  \item All data is kept in appropriate structures in main memory AND fetched when the data is queried initially
  \item De-serialized from JSON at the earliest convenience in the ESDM layer
  \item Backend communicates via JSON to ESDM layer
\end{itemize}

Callgraph from user perspective:

\begin{itemize}
  \item \lstinline|c = container_open("name")| \textless= here we read the metadata for "name" and generate appropriate structures
  \item \lstinline|d = dataset_open("dset")| \textless= here we read the metadata for "dset" and generate appropriate structures
  \item \lstinline|read(d)|
  \item \lstinline|dataset_destroy(d)|
  \item \lstinline|container_destroy(c)|
\end{itemize}

Alternative workflow with unkown dsets:

\begin{itemize}
  \item \lstinline|c = container_open("name")| \textless= here we read the metadata for "name" and generate appropriate structures
  \item \lstinline|it = dataset_iterator()|
  \item \lstinline|for x in it:|
    \begin{itemize}
      \item \lstinline|dataset_iterator_dataset(x)|
    \end{itemize}
  \item \lstinline|read(d)|
  \item \lstinline|dataset_destroy(d)|
  \item \lstinline|container_destroy(c)|
\end{itemize}
