\section{Styleguide for ESDM development}%
\label{styleguide-for-esdm-development}

This document describes the style guide to use in the code.

\subsection{General rules}\label{general-rules}

\begin{itemize}
  \item Do not break the line after a fixed number of characters as this is the duty of the editor to use some softwrap.
    \begin{itemize}
      \item You may use a line wrap, if that increases readability (see example below).
    \end{itemize}
  \item Use two characters for indentation per level
  \item Documentation with Doxygen needs to be added on the header files
  \item Ensure that the code does not produce WARNINGS
  \item Export only the functions to the user that is needed by the user
  \item The private (module-internal) interface is defined in \textbf{internal}.h
\end{itemize}

\subsection{Naming conventions}\label{naming-conventions}

\begin{itemize}
  \item use lower case for the public interface
  \item functions for users provided by ESDM start with esdm\_
  \item auxiliary functions that are used internally start with ea\_ (ESDM auxiliary) and shall be defined inside esdm-internal.h
\end{itemize}

\subsection{Example Code}\label{example-code}

\begin{lstlisting}
//First add standard libraries
#include <stdio.h>
#include <stdlib.h>

// Add an empty line before adding any ESDM include file
#include <esdm-internal.h>

struct x_t{
  int a;
  int b;
  int *p;
};

// needs always to be split separately, 
// to allow it to coexist in a public header file
typedef struct x_t x_t; 

int testfunc(int a){
  {
    // Additional basic block
  }
  if (a == 5){

  }else{

  }

  // allocate variables as late as possible, 
  // such that we can see when it is needed and what it does.
  int ret;

  switch(a){
    case(5):{
      break;
    }case(2):{

    }
    default:{
      xxx
    }
  }

  return 0;
}
\end{lstlisting}
