\section{Grid deduplication}
\subsection{Requirements}

\begin{itemize}
  \item The user must be allowed to create throw-away grids for reading
  \item It must be avoided that there are several fragments with the exact same shape
  \item It must be avoided that there are several grids with the exact same axis and subgrid structure
  \item[$\rightarrow$] deduplication must happen on two distinct levels: the fragment level and the grid level
  \item It is not possible to reference count grids or fragments, as these objects contain a reference to their owning dataset, and thus must not survive its destruction
    \begin{itemize}
      \item[$\rightarrow$] deduplication must rely on proxy objects that reference the already existing objects
    \end{itemize}
  \item User code can hold pointers to subgrids
    \begin{itemize}
      \item[$\rightarrow$] grids cannot be replaced with proxy objects, they must be turned into proxy objects
      \item[$\rightarrow$] grids that are turned into proxy objects must retain their subgrid structure to ensure proper destruction
    \end{itemize}
  \item A fragment cannot be matched with a subgrid, and a subgrid cannot be matched with another subgrid that differs in one of the axes or subgrids
    \begin{itemize}
      \item[$\rightarrow$] deduplication cannot happen while the grid is still in fixed axes state
      \item[$\rightarrow$] \lstinline|esdm_read_grid()| and \lstinline|esdm_write_grid()| should be defined to put a grid into fixed structure state, allowing deduplication on their first call
    \end{itemize}
\end{itemize}



\subsection{Implementation}
\begin{itemize}
  \item Fragments are managed in a centralized container (hashtable keyed by their extends), and referenced by the grids.
    \begin{itemize}
      \item A count of referencing grids can be added if necessary.
        This is not a general ref count, the dataset still owns the fragments and destructs them in its destructor.
        The grid count would allow the dataset to delete a fragment when its containing grids get deleted.
      \item The fragment references in the grids remain plain pointers, but the grids loose their ownership over the fragments.
      \item The MPI code that handles grids must be expanded to also communicate fragment information explicitly.
      \item As a side effect, this also allows the dataset to unload any fragments to prevent run-away memory consumption.
    \end{itemize}
  \item Grids contain a delegate pointer.
    \begin{itemize}
      \item All grid methods must first resolve any delegate chain.
      \item The delegate pointer is set when the grid is first touched with an I/O or MPI call and detected to be structurally identical to an existing grid.
      \item When the delegate pointer is set, the delegate pointers of all subgrids are also set.
      \item There are two possible implementations for managing grid proxy objects:
        \begin{itemize}
          \item The proxy grid's structure data remains valid (axes and cell matrix), the proxy's destructor recursively destructs its subgrid proxies.
          \item All existing grids are managed via a flat list of grids within the dataset, and grids that become proxies get their axis and matrix data deleted immediately.
        \end{itemize}
    \end{itemize}
\end{itemize}



\subsection{Roadmap for Implementation}

\begin{enumerate}
 \item Create a hash function that works on hypercubes and offset/size arrays.
 \item Centralize the storage of fragments in a hash table.
    This deduplicates fragments, and takes fragment ownership away from grids.
    This will break the MPI code.
 \item Fix the MPI interface by communicating fragment metadata separately from grid metadata.
 \item Centralize the storage of grids in the dataset.
    The dataset should have separate lists for complete top-level grids, incomplete top-level grids, and subgrids.
 \item Implement delegates for grids.
 \item Implement the grid matching machinery to create the delegates.
\end{enumerate}
 
